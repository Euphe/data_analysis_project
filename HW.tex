\documentclass[a4paper,14pt]{article}

%\includeonly{topics/t2,topics/t3} % компилировать только указанные главы
%%% Работа с русским языком
\usepackage{cmap}					% поиск в PDF
\usepackage{mathtext} 				% русские буквы в формулах
\usepackage[T2A]{fontenc}			% кодировка
\usepackage[utf8]{inputenc}			% кодировка исходного текста
\usepackage[english,russian]{babel}	% локализация и переносы
\frenchspacing
%\usepackage{fontspec} 
%\setmainfont[Ligatures={TeX,Historic}]{Times New Roman}

%%%\usepackage{fancyhdr} % {\tiny  }Колонтитулы 
%\pagestyle{fancy} 
%\renewcommand{\headrulewidth}{0mm} % Толщина линейки, отчеркивающей верхний колонтитул 
%\lfoot{Нижний левый} 
%\rfoot{Нижний правый} 
%\rhead{Верхний правый} 
%\chead{Верхний в центре} 
%\lhead{Верхний левый} 
% \cfoot{Нижний в центре} % По умолчанию здесь номер страницы 

%%% Дополнительная работа с математикой
\usepackage{amsmath,amsfonts,amssymb,amsthm,mathtools} % AMS
\usepackage{icomma} % "Умная" запятая: $0,2$ --- число, $0, 2$ --- перечисление
%\usepackage{ dsfont } % more math fonts!
%\renewcommand{\epsilon}{\ensuremath{\varepsilon}}
%\renewcommand{\phi}{\ensuremath{\varphi}}
%\renewcommand{\kappa}{\ensuremath{\varkappa}}
%\renewcommand{\le}{\ensuremath{\leqslant}}
\renewcommand{\leq}{\ensuremath{\leqslant}}
%\renewcommand{\ge}{\ensuremath{\geqslant}}
\renewcommand{\geq}{\ensuremath{\geqslant}}
%\renewcommand{\emptyset}{\varnothing}

%% Номера формул
%\mathtoolsset{showonlyrefs=true} % Показывать номера только у тех формул, на которые есть \eqref{} в тексте.
%\usepackage{leqno} % Нумереация формул слева

%%% Работа с картинками
\usepackage{graphicx}  % Для вставки рисунков
\graphicspath{crimeimages/}  % папки с картинками
\setlength\fboxsep{3pt} % Отступ рамки \fbox{} от рисунка
\setlength\fboxrule{1pt} % Толщина линий рамки \fbox{}
\usepackage{wrapfig} % Обтекание рисунков текстом

%%% Работа с таблицами
\usepackage{array,tabularx,tabulary,booktabs} % Дополнительная работа с таблицами
\usepackage{longtable}  % Длинные таблицы
\usepackage{multirow} % Слияние строк в таблице

%% Свои команды
\DeclareMathOperator{\sgn}{\mathop{sgn}}

%% Перенос знаков в формулах (по Львовскому)
\newcommand*{\hm}[1]{#1\nobreak\discretionary{}
	{\hbox{$\mathsurround=0pt #1$}}{}}

\newcommand{\source}[1]{\small{Источник: #1}}

%%% Страница
\usepackage{extsizes} % Возможность сделать 14-й шрифт
\usepackage{geometry} % Простой способ задавать поля
\geometry{top=20mm}
\geometry{bottom=20mm}
\geometry{left=35mm}
\geometry{right=15mm}
%
%\usepackage{fancyhdr} % Колонтитулы
% 	\pagestyle{fancy}
%\renewcommand{\headrulewidth}{0pt}  % Толщина линейки, отчеркивающей верхний колонтитул
% 	\lfoot{Нижний левый}
% 	\rfoot{Нижний правый}
% 	\rhead{Верхний правый}
% 	\chead{Верхний в центре}
% 	\lhead{Верхний левый}
%	\cfoot{Нижний в центре} % По умолчанию здесь номер страницы

\usepackage{setspace} % Интерлиньяж
\onehalfspacing % Интерлиньяж 1.5
%\doublespacing % Интерлиньяж 2
%\singlespacing % Интерлиньяж 1

\usepackage{lastpage} % Узнать, сколько всего страниц в документе.
%\usepackage{soul} % Модификаторы начертания

\usepackage{hyperref}
\usepackage[usenames,dvipsnames,svgnames,table,rgb]{xcolor}
\hypersetup{				% Гиперссылки
	unicode=true,           % русские буквы в раздела PDF
	%	pdftitle={Заголовок},   % Заголовок
	%	pdfauthor={Автор},      % Автор
	%	pdfsubject={Тема},      % Тема
	%	pdfcreator={Создатель}, % Создатель
	%	pdfproducer={Производитель}, % Производитель
	%	pdfkeywords={keyword1} {key2} {key3}, % Ключевые слова
	colorlinks=true,       	% false: ссылки в рамках; true: цветные ссылки
	linkcolor=blue,          % внутренние ссылки
	citecolor=blue,        % на библиографию
	filecolor=magenta,      % на файлы
	urlcolor=ForestGreen           % на URL
}


% \usepackage{csquotes} % Еще инструменты для ссылок
%\usepackage[style=authoryear,maxcitenames=2,backend=biber,sorting=nty]{biblatex}
\usepackage{multicol} % Несколько колонок

%%% Программирование
\usepackage{etoolbox} % логические операторы
%\usepackage[style=authoryear,maxcitenames=2,backend=biber,sorting=nty]{biblatex}

\usepackage{tikz} % Работа с графикой
% \usepackage{pgfplots}
% \usepackage{pgfplotstable}

% % % Для работы с report
%\renewcommand{\chaptername}{Глава}


\usepackage{cite} % Работа с библиографией
%\usepackage[superscript]{cite} % Ссылки в верхних индексах
%\usepackage[nocompress]{cite} % 
%\usepackage{csquotes} % Еще инструменты для ссылок

% \usepackage{natbib} %библиография плюс https://ru.wikibooks.org/wiki/LaTeX/%D0%A3%D0%BF%D1%80%D0%B0%D0%B2%D0%BB%D0%B5%D0%BD%D0%B8%D0%B5_%D0%B1%D0%B8%D0%B1%D0%BB%D0%B8%D0%BE%D0%B3%D1%80%D0%B0%D1%84%D0%B8%D0%B5%D0%B9
%\bibliographystyle{rmpaps} 


%%%% Работа со списками
%\usepackage{enumitem} %%% дополнительная работа со списками. !а может и неправильно написано — гугли
\usepackage{paralist}										% compact lists

\renewcommand{\citeleft}{(}
\renewcommand{\citeright}{)}

\usepackage{adjustbox} 
\addto\captionsrussian{% Replace "english" with the language you use
	\renewcommand{\contentsname}%
	{Table Of Contents}%
}

\author{Yulia Gurova}
\title{Tax systems}



\begin{document}
	
	\thispagestyle{empty}    % +1 - это титульный лист
	
	\begin{center}
		Правительство Российской Федерации \\
		Федеральное государственное автономное образовательное учреждение \\ высшего профессионального образования \\ «Национальный исследовательский университет \\ «Высшая школа экономики»
		
		\large
		\vspace{2 cm}
		%	\textsf{
	%	Факультет экономических наук
	%	\\ Образовательная программа «Экономика» %}           %ТЕКСТ БЕЗ ЗАСЕЧЕК
	\end{center}
	
	\vspace{2 cm}
	\begin{center}
		%	\vspace{13ex}
		{\textbf{Анализ количества убийств в фильмах} \\\vspace{1 cm} \textbf{The Analysis of On-screen Movie Kills} \\ \vspace{0.5 cm} Homework Project 2018/2019 } 
		
		
	\end{center}
	
	\vspace{2 cm}
	
	\begin{flushright}
		%	\noindent
		{ \textbf{Команда:} \\ Борис Цейтлин \\ Константин Ромащенко \\ Юлия Гурова}
		
		\vspace{1 cm}
		
		{ \textbf{1 курс магистратуры } \\ Программа: «Науки о данных» \\
			Факультет компьютерных наук }
	\end{flushright}
	
	\begin{center}
		\vfill
		Москва 2018	
	\end{center}
	

	
	\newpage


	\thispagestyle{empty}    % +1 - это титульный лист

\begin{center}
	

THE RUSSIAN GOVERNMENT \\
FEDERAL STATE AUTONOMUS EDUCATIONAL INSTITUTION \\ FOR HIGHER PROFESSIONAL EDUCATION \\ NATIONAL RESEARCH UNIVERSITY \\ ``HIGHER SCHOOL OF ECONOMICS''
	
	\large
	\vspace{2 cm}
	%	\textsf{
	%	Факультет экономических наук
	%	\\ Образовательная программа «Экономика» %}           %ТЕКСТ БЕЗ ЗАСЕЧЕК
\end{center}

\vspace{2 cm}
\begin{center}
	%	\vspace{13ex}
	 \vspace{1 cm} \textbf{The Analysis of On-screen Movie Kills} \\ \vspace{0.5 cm} Homework Project 2018/2019 
	
	
\end{center}

\vspace{2 cm}

\begin{flushright}
	%	\noindent
	{ \textbf{The team:} \\ Boris Tseitlin \\ Konstantin Romashchenko \\ Yulia Gurova}
	
	\vspace{1 cm}
	
	{ \textbf{MSc Program “Data Science” } \\ 1$^{st}$ year \\
		Faculty of Computer Science }
\end{flushright}

\begin{center}
	\vfill
	Moscow 2018	
\end{center}

	\newpage
\tableofcontents
	\newpage
	
	
\section{THE CHOICE OF THE DATASET}

The dataset that is used for the project contains information about  on-screen deaths in movies. There are 545 movies (more then 100 objects) and 8 characteristics including names, so the dataset meets the requirements. 

The sourse for the data is the thematic web-site \href{http://www.moviebodycounts.com/}{moviebodycounts.com}. This dataset was processed and published on \href{https://figshare.com/articles/On_screen_movie_kill_counts_for_hundreds_of_films/889719}{figshare.com}. It was gathered  in accordance with the \href{http://moviebodycounts.proboards.com/thread/6}{rules}, which are published on the web-site. We took several characteristics for the consideration: the year of the film, MPAA rating (Motion Picture Association of America film rating system), genre or genres, the name of the director, the lenth of the film in minutes,  IMDB rating based on user ratings. The main feature that we consider is the number of on-screen deaths in the movie. 

The analysis of the data may reveal how the ratings depend on the number of deaths, how this  number is changing with the year of the release and so on. Moreover, genre and length of the film combined with the violence on the screen may give the idea of the age ratings. This is a good set for the classification and clustering problems, as the films are grouped on genres, MPAA ratings which are similar inside the groups, but dissimilar between them. 

This study analysis may be the first step to the automation of age rating systems. Moreover it may be helpful in the development of the recommendation systems. And, as watching movies is the common interest of our team, the work with the dataset will inspire us to further conquest in Data Analysis.


\newpage

\section{K-means clustering}
\section{Cluster Interpretation}
\section{Contingency Table Analysis}
\section{PCA: Hidden Factor \& Data visualization}
\section{2D regression}


\end{document}